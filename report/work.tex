%% -*- coding: utf-8 -*-
\documentclass[12pt,a4paper]{scrartcl} 
\usepackage[utf8]{inputenc}
\usepackage[english,russian]{babel}
\usepackage{indentfirst}
\usepackage{misccorr}
\usepackage{graphicx}
\usepackage{amsmath}
\begin{document}
\section{Введение}
\label{sec:intro}

Создать программу для генерации случайных паролей заданной длины и сложности

\section{Ход работы}
\label{sec:exp}

\subsection{Код приложения}
\label{sec:exp:code}
\begin{verbatim}
#include <iostream>
#include <string>
#include <cstdlib>
#include <ctime>

using namespace std;

string generatePassword(int length) {
    string password;
    static const char alphanum[] = // Массив из набора символов
        "0123456789"
        "ABCDEFGHIJKLMNOPQRSTUVWXYZ"
        "abcdefghijklmnopqrstuvwxyz"
        "!@#$%^&*";
    int stringLength = sizeof(alphanum) - 1;

    for (int i = 0; i < length; ++i) {
        password += alphanum[rand() % stringLength];
    }

    return password;
}

int main() {
    setlocale(LC_ALL, "Russian"); // Русский язык
    srand(time(0)); // Инициализация генератора случайных чисел

    int passwordLength;
    cout << "Введите длину пароля: ";
    cin >> passwordLength;

    string password = generatePassword(passwordLength);
    cout << "Сгенерированный пароль: " << password << endl;

    return 0;
}
\end{verbatim}

\subsection{Пример работы}
\label{sec:example}
Пример работы представлен на рис.~\ref{fig:enter-label}
\begin{figure}[h]
    \centering
    \includegraphics[width=1\textwidth]{example.png}
    \caption{Пример работы}\label{fig:enter-label}
\end{figure}

\end{document}

