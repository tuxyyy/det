%% -*- coding: utf-8 -*-
\documentclass[12pt,a4paper]{scrartcl} 
\usepackage[utf8]{inputenc}
\usepackage[english,russian]{babel}
\usepackage{indentfirst}
\usepackage{misccorr}
\usepackage{graphicx}
\usepackage{amsmath}
\begin{document}
\section{Введение}
\label{sec:intro}

Определителем квадратной матрицы
$ A_{m,n} = 
 \begin{pmatrix}
  a_{11} & a_{12} \\
  a_{21} & a_{22} \\
 \end{pmatrix}
$ второго порядка называется число 
$|A|= {a_{11} a_{22}-a_{12} a_{21}}$.
Определителем 
$ A = 
 \begin{pmatrix}
  a_{11} & \cdots & a_{1n} \\
  \vdots  & \ddots & \vdots  \\
  a_{n1}  & \cdots & a_{nn} 
 \end{pmatrix}
$ квадратной матрицы порядка $n, n\geq3$, называется число $|A|=\displaystyle \sum_{k=1}^n(-1)^{k+1}a_{1k}M_k$, где $M_k$ - определитель матрицы порядка $n-1$, полученной из матрицы A вычеркиванием первой строки и столбца с номером k.

Пример приведен в пункте ~\ref{sec:exp} на стр.~\pageref{sec:example}.

\section{Ход работы}
\label{sec:exp}

\subsection{Код приложения}
\label{sec:exp:code}
\begin{verbatim}
#include <iostream>
#include <vector>


using namespace std;


// Функция для вывода матрицы
void printMatrix(vector<vector<double>> matrix) 
{
    for (int i = 0; i < matrix.size(); i++) 
    {
        for (int j = 0; j < matrix[i].size(); j++)
        {
            cout << matrix[i][j] << " ";
        }
        cout << endl;
    }
}


// Функция для нахождения определителя матрицы
double determinant(vector<vector<double>> matrix) 
{
    double det = 1;
    int n = matrix.size();

    // Приводим матрицу к треугольному виду
    for (int i = 0; i < n; i++) 
    {
        for (int j = i + 1; j < n; j++) 
        {
            double factor = matrix[j][i] / matrix[i][i];
            for (int k = i; k < n; k++) 
            {
                matrix[j][k] -= factor * matrix[i][k];
            }
        }
    }

    // Вычисляем определитель как произведение элементов на главной диагонали
    for (int i = 0; i < n; i++) 
    {
        det *= matrix[i][i];
    }

    return det;
}


int main() 
{
    setlocale(LC_ALL, "Russian");
    // Вводим размерность матрицы
    int n;
    cout << "Введите размер матрицы: ";
    cin >> n;


    // Создаем матрицу
    vector<vector<double>> matrix(n, vector<double>(n));
    cout << "Введите элементы матрицы:" << endl;
    for (int i = 0; i < n; i++) 
    {
        for (int j = 0; j < n; j++) 
        {
            cin >> matrix[i][j];
        }
    }

    // Выводим матрицу
    cout << "Матрица:" << endl;
    printMatrix(matrix);

    // Вычисляем определитель
    double det = determinant(matrix);

    // Выводим определитель
    cout << "Определитель равен: " << det << endl;

    return 0;
}
\end{verbatim}

\subsection{Пример работы}
\label{sec:example}
Пример работы представлен на рис.~\ref{fig:wex}
\begin{figure}[h]
	\centering
	\includegraphics[width=1\textwidth]{example.jpg}
	\caption{Пример работы}\label{fig:wex}
\end{figure}

\end{document}